
% Remove green borders around links
\usepackage{xcolor}
\usepackage{hyperref}
\xdefinecolor{blue}{rgb}{0.28,0.65,0.9}
\definecolor{grey}{rgb}{0.45, 0.45, 0.45}
\hypersetup{
    colorlinks,
    linkcolor={blue},
    citecolor={blue},
    urlcolor={blue}
}
\usepackage{setspace}
\usepackage{multicol}


% need this for the chunk captioning
\usepackage[margin=0.6cm]{caption}
\usepackage{floatrow}
\usepackage{subcaption}
\DeclareNewFloatType{chunk}{placement=H, fileext=chk, name=}
\captionsetup{options=chunk}
\renewcommand{\thechunk}{Code Block~\thesection.\arabic{chunk}}
\makeatletter
\@addtoreset{chunk}{section}
\makeatother

% % If you need to figure out what font you're using, this can be useful:
% \makeatletter
% \edef\textFontName{\fontname\csname
%   \f@encoding/\f@family/\f@series/\f@shape/\f@size\endcsname}
% \edef\mathFontName{\fontname\textfont0}
% \edef\mathLetterFontName{\fontname\textfont1}
% \makeatother
% % Put  \textFontName in document to see font. 
% % From https://tex.stackexchange.com/questions/471070/is-there-a-latex-command-to-print-the-name-of-the-font-used-for-text-and-for-mat